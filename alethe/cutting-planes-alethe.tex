\documentclass[12pt]{article}

% https://canizo.org/templates -> article

\usepackage[utf8]{inputenc}
\usepackage[T1]{fontenc}
\usepackage[a4paper, margin=2.7cm]{geometry}
\usepackage{amsmath, amsthm, amsfonts, amssymb}
\usepackage{mathrsfs}           % \mathscr font.
\usepackage[colorlinks=true,linkcolor=blue,citecolor=blue,urlcolor=blue,breaklinks]{hyperref}
\usepackage{bbm} 
\usepackage{breakurl}
\usepackage{natbib}
\usepackage{url}
\bibliographystyle{plainnat-linked}

\usepackage{xcolor}
\definecolor{light-gray}{gray}{0.95}
\newcommand{\code}[1]{\colorbox{light-gray}{\texttt{#1}}}

% Paper title and author
\title{\textbf{The Alethe Proof Format}}
\author{Bernardo Borges \and Haniel Barbosa}

\begin{document}

\maketitle

\begin{abstract}
    Specifying more rules for alethe in the Cutting Planes reasoning
\end{abstract}

\tableofcontents

\section{Introduction}
In this article we describe how the cutting planes reasoning can be expressed in the alethe
proof format, in order to make use of checkers (carcara).

\section{Pseudo Boolean Inequality in Alethe}
The Pseudo Boolean format consists of:
\begin{equation}
    \sum_i{a_i l_i} \ge A
\end{equation}
where,
\begin{equation}
    \begin{gathered}
        A, a_i \in \mathbb{N} \\
        l_i \in \{ x_i, \overline x_i \}, \qquad x_i + \overline x_i = 1
    \end{gathered}
\end{equation}

In order to express it in alethe, we define an expression:

\centerline{\code{(>= (+ <TERMS> 0) <A>)}}

where \code{<TERMS>} is a list of either:
\begin{enumerate}
    \item \code{(* <a\_i> <l\_i>)} in a plain literal $a_i x_i$.
    \item \code{(* (- 1 <a\_i>) <l\_i>)} in a negated literal $a_i \overline x_i$
\end{enumerate}

and \code{<A>} is the natural constant $A$.

\section{Cutting Planes Rules in Alethe}

\subsection{Rule 1: cp\_addition}
We can apply the rules, using the `step' syntax:

\begin{verbatim}
(assume c1 (>= (+ (* 2 x1) 0) 1))
(assume c2 (>= (+ (* 1 (- 1 x1)) 0) 1))
(step t1 (cl (>= (+ (* 1 x1) 0) 1))
    :rule cp\_addition :premises (c1 c2)
)
\end{verbatim}

\subsection{Rule 2: cp\_multiplication}
\begin{verbatim}
(assume c1 (>= (+ (* 1 x1) 0) 1))
(step t1 (cl (>= (+ (* 2 x1) 0) 2))
    :rule cp_multiplication :premises (c1) :args (2)
)
\end{verbatim}

\subsection{Rule 3: cp\_division}
\begin{verbatim}
(assume c1 (>= (+ (* 2 x1) 0) 2))
(step t1 (cl (>= (+ (* 1 x1) 0) 1))
    :rule cp_division :premises (c1) :args (2)
)
\end{verbatim}

\subsection{Rule 4: cp\_saturation}
\begin{verbatim}
(assume c1 (>= (+ (* 2 x1) 0) 1))
(step t1 (cl (>= (+ (* 1 x1) 0) 1))
    :rule cp_saturation :premises (c1)
)
\end{verbatim}

\bibliography{bibliography}

\end{document}

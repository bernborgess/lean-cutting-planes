\subsection{Pseudo Boolean Reasoning}

Pseudo-Boolean constraints are mathematical expressions involving binary (0 or 1) variables
(or literals) and integer coefficients, typically in the form of inequalities.
Specifically, a pseudo-Boolean constraint is an inequality of the following form:
\[
    \sum_i a_i \cdot l_i \geq A
\]
where $A$ is called \textbf{constant}, $a_i$ are \textbf{coefficients},
and $l_i$ are \textbf{literals}, that are either:
\begin{itemize}
    \item \textbf{plain} literal, a term \inlineAlethe{x};
    \item \textbf{negated} literal, a term of the form \inlineAlethe{(- 1 x)}
\end{itemize}

where the \inlineAlethe{x} value is a pseudo-boolean variable,
i.e. it will resolve to values \inlineAlethe{0} (false) or \inlineAlethe{1} (true).
All these values are of sort \inlineAlethe{Int}.

To form a summation we use a list of added terms of form,
\inlineAlethe{(+ <T1> <T2> ... 0)} and each term is
\inlineAlethe{(* a_i <L1>)}, with a coefficient and a literal, always ending with a \inlineAlethe{0}.


\paragraph{Normalized Form}
A pseudo-Boolean constraint is in the \textbf{normalized form} if:
\begin{itemize}
    \item All \textbf{coefficients} $a_i$ are non-negative;
    \item The \textbf{constant} $A$ is non-negative;
\end{itemize}

%This form is a precondition for the \proofRule{cp_division} and \proofRule{cp_saturation} rules.
This form is a precondition for the \texttt{cp\_division} and \texttt{cp\_saturation} rules.

\subsection{Bitvector Reasoning with Pseudo-Boolean Bitblasting}
An alternative approach to do bitvector reasoning is to perform bitblasting with Pseudo-Boolean expressions, which
are more expressive in terms of conciseness for some operations.

Similarly to regular bitblasting, the Alethe calculus uses
multiple families of helper functions to express pseudo-boolean bitblasting:
$\lsymb{pbbT}$, $\lsymb{intOf}_m$, $\lsymb{bvsize}$, and $\lsymb{bv}_n^i$.
The last two functions work the same as in regular bitblasting, where the first two introduce the different
underlying pseudo-boolean values, which are represented as values of \inlineAlethe{Int}.

The family $\lsymb{pbbT}$ consists of one function for each bitvector sort
$(\lsymb{BitVec}\;n)$.
\[
    \lsymb{pbbT}\,:\,\underbrace{\lsymb{Int}\,\dots\,\lsymb{Int}}_n\;(\lsymb{BitVec}\;n).
\]

\noindent
The function $\lsymb{pbbT}$ takes a list of pseudo-booleans i.e. integers that must be
\inlineAlethe{0} or \inlineAlethe{1} arguments and packs them into a bitvector.

The functions $\lsymb{intOf}_m$ are the inverse of $\lsymb{pbbT}$.  They extract
a bit of a bitvector as a boolean.  Just as the built in $\lsymb{extract}$
symbol, $\lsymb{intOf}_m$ is used as an indexed symbol.  Hence, for $m \leq n$, we
write \inlineAlethe{(_ @intOf} $m$ \inlineAlethe{)}, to denote functions
\[
    \lsymb{intOf}_m : (\lsymb{BitVec}\;n) \to \lsymb{Int}.
\]
These functions are defined as
\[
    \lsymb{intOf}_m \langle u_1, \dots, u_n \rangle := u_m.
\]

All other concepts not related with these rules will use the same definitions of propositional
bitblasting.

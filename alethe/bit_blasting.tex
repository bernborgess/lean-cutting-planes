\begin{RuleDescription}{pbblast_bveq}
    Consider bitvectors \textbf{x} and \textbf{y} of length $n$.
    The pseudo-boolean bitblasting of their equality is expressed by:

    \begin{AletheX}
        $i$. & \ctxsep & $(\lsymb{=}\ x\ y) \approx A$ & (\currule)
    \end{AletheX}
    The term ``$A$'' is the conjunction of PseudoBoolean constraints:

    \[ \bigwedge_{i=0}^{n-1}{\left(x_i-y_i = 0\right)} \]

    \noindent
    Or similarly, using a single constraint:

    \[ \sum_{i=0}^{n-1}{2^i x_{n-i-1}} - \sum_{i=0}^{n-1}{2^i y_{n-i-1}} = 0\]

\end{RuleDescription}

\begin{RuleDescription}{pbblast_bvult}
    The `unsigned-less-than' operation over BitVectors with $n$ bits is expressed using PseudoBoolean inequalities by:

    \begin{AletheX}
        $i$. & \ctxsep & $(\lsymb{bvult}\ x\ y) \approx A$ & (\currule)
    \end{AletheX}
    The term ``$A$'' is `true' iff:

    \[
        \sum_{i=0}^{n-1} 2^i\mathbf{y}_{n-i-1} - \sum_{i=0}^{n-1} 2^i\mathbf{x}_{n-i-1} \ge 1.
    \]

\end{RuleDescription}


\begin{RuleDescription}{bv_not_equal}
    The \currule{} rule, meaning bitwise inequality, can be concisely expressed by:
    \[
        \mathbf{z} = \text{XOR}(x, y); \qquad \sum_{i=0}^{k-1} \mathbf{z}_i \geq 0
    \]
\end{RuleDescription}

\begin{RuleDescription}{bv_not_ult}
    The negation of the \currule{} rule, meaning $x \geq y$ for unsigned integers, can be expressed as:
    \[
        \neg ULT(x, y) \equiv UGE(x, y)
    \]
\end{RuleDescription}

\begin{RuleDescription}{bv_uge}
    The \currule{} rule for unsigned greater than or equal, $x \geq y$, is defined as:
    \[
        \sum_{i=0}^{k-1} 2^i\mathbf{x}_{k-i-1} - \sum_{i=0}^{k-1} 2^i\mathbf{y}_{k-i-1} \geq 0.
    \]
\end{RuleDescription}

\begin{RuleDescription}{bv_not_uge}
    The negation of the \currule{} rule, meaning $x < y$ for unsigned integers, is expressed by:
    \[
        \neg UGE(x, y) \equiv ULT(x, y)
    \]
\end{RuleDescription}

\begin{RuleDescription}{bv_ule}
    The \currule{} rule for unsigned less than or equal, $x \leq y$, is equivalent to:
    \[
        ULE(x, y) \equiv UGE(y, x)
    \]
\end{RuleDescription}

\begin{RuleDescription}{bv_not_ule}
    The negation of the \currule{} rule, meaning $x > y$, is expressed by:
    \[
        \neg ULE(x, y) \equiv UGT(x, y)
    \]
\end{RuleDescription}

\begin{RuleDescription}{bv_ugt}
    The \currule{} rule for unsigned greater than, $x > y$, is equivalent to:
    \[
        UGT(x, y) \equiv ULT(y, x)
    \]
\end{RuleDescription}

\begin{RuleDescription}{bv_not_ugt}
    The negation of the \currule{} rule, meaning $x \leq y$, is expressed by:
    \[
        \neg UGT(x, y) \equiv ULE(x, y)
    \]
\end{RuleDescription}

\begin{RuleDescription}{bv_pbblast_step_slt}
    The `slt' operation over BitVectors with $n$ bits is expressed using PseudoBoolean inequalities by:

    \begin{AletheX}
        $i$. & \ctxsep & $(\lsymb{bvslt}\ x\ y) ≈ A$ & (\currule) \\
    \end{AletheX}
    The term ``$A$'' is `true' iff:

    \[
        2^{n-1} \overline{y_0} + \sum_{i=0}^{n-2} 2^i\mathbf{y}_{n-i-1} + 2^{n-1}x_0 + \sum_{i=0}^{n-2} 2^i\mathbf{\overline{x}}_{n-i-1} \ge 2^{n}.
    \]
\end{RuleDescription}

\begin{RuleDescription}{bv_not_slt}
    The negation of the \currule{} rule, meaning $x \geq y$ for signed integers, is expressed by:
    \[
        \neg SLT(x, y) \equiv SGE(x, y)
    \]
\end{RuleDescription}

\begin{RuleDescription}{bv_sge}
    The \currule{} rule for signed greater than or equal, $x \geq y$, is defined as:
    \[
        -(2^{k-1})\mathbf{x}_0 + \sum_{i=0}^{k-2} 2^i\mathbf{x}_{k-i-1} + 2^{k-1}\mathbf{y}_0 - \sum_{i=0}^{k-2} 2^i\mathbf{y}_{k-i-1} \geq 0.
    \]
\end{RuleDescription}

\begin{RuleDescription}{bv_not_sge}
    The negation of the \currule{} rule, meaning $x < y$ for signed integers, is expressed by:
    \[
        \neg SGE(x, y) \equiv SLT(x, y)
    \]
\end{RuleDescription}

\begin{RuleDescription}{bv_sle}
    The \currule{} rule for signed less than or equal, $x \leq y$, is equivalent to:
    \[
        SLE(x, y) \equiv SGE(y, x)
    \]
\end{RuleDescription}

\begin{RuleDescription}{bv_not_sle}
    The negation of the \currule{} rule, meaning $x > y$, is expressed by:
    \[
        \neg SLE(x, y) \equiv SGT(x, y)
    \]
\end{RuleDescription}

\begin{RuleDescription}{bv_sgt}
    The \currule{} rule for signed greater than, $x > y$, is equivalent to:
    \[
        SGT(x, y) \equiv SLT(y, x)
    \]
\end{RuleDescription}

\begin{RuleDescription}{bv_not_sgt}
    The negation of the \currule{} rule, meaning $x \leq y$, is expressed by:
    \[
        \neg SGT(x, y) \equiv SLE(x, y)
    \]
\end{RuleDescription}

\begin{RuleDescription}{pbblast_step_bvand}
    The `bvand' operation over BitVectors with n bits is expressed using PseudoBoolean inequalities by:

    \begin{AletheX}
        $i$. & \ctxsep & $(\lsymb{bvand}\ x\ y) \approx [r_0,\dots,r_1] \land A$ & (\currule) \\
    \end{AletheX}\\
    The term ``$A$'' is the conjunction of these PseudoBoolean inequalities, for $0 \le i < n$:

    \[ x_i-r_i\ge 0 \]
    \[ y_i-r_i\ge 0 \]
    \[ r_i-x_i-y_1\ge -1 \]

\end{RuleDescription}

\newpage
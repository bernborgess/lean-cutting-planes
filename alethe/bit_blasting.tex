\begin{RuleDescription}{pbblast_bveq}
    Consider bitvectors \textbf{x} and \textbf{y} of length $n$.
    The pseudo-boolean bitblasting of their equality is expressed by:

    \begin{AletheX}
        $i$. & \ctxsep & $(\lsymb{=}\ x\ y) \approx A$ & (\currule)
    \end{AletheX}
    The term ``$A$'' is the PseudoBoolean constraint:

    \[ \sum_{i=0}^{n-1}{2^i x_{i}} - \sum_{i=0}^{n-1}{2^i y_{i}} = 0\]

\end{RuleDescription}

\begin{RuleDescription}{pbblast_bvult}
    The `unsigned-less-than' operation over BitVectors with $n$ bits is expressed using PseudoBoolean inequalities by:

    \begin{AletheX}
        $i$. & \ctxsep & $(\lsymb{bvult}\ x\ y) \approx A$ & (\currule)
    \end{AletheX}
    The term ``$A$'' is `true' iff:

    \[
        \sum_{i=0}^{n-1} 2^i\mathbf{y}_{i} - \sum_{i=0}^{n-1} 2^i\mathbf{x}_{i} \ge 1.
    \]

\end{RuleDescription}

\begin{RuleDescription}{pbblast_bvugt}
    The `unsigned-greater-than' operation over BitVectors with $n$ bits is expressed using PseudoBoolean inequalities by:

    \begin{AletheX}
        $i$. & \ctxsep & $(\lsymb{bvugt}\ x\ y) \approx A$ & (\currule)
    \end{AletheX}
    The term ``$A$'' is `true' iff:

    \[
        \sum_{i=0}^{n-1} 2^i\mathbf{x}_{i} - \sum_{i=0}^{n-1} 2^i\mathbf{y}_{i} \ge 1.
    \]

    \noindent
    Or in terms of \proofRule{pbblast_bvult}:

    \begin{AletheX}
        $i$. & \ctxsep & $(\lsymb{bvugt}\ x\ y) \approx (\lsymb{bvult}\ y\ x)$ & (\currule)
    \end{AletheX}
\end{RuleDescription}


% https://github.com/cvc5/cvc5/blob/96a35d7cc97ee375e263ab43a2ed9ba03cc32858/src/rewriter/node.py#L44
\begin{RuleDescription}{pbblast_bvuge}
    The `unsigned-greater-or-equal' operation over BitVectors with $n$ bits is expressed using PseudoBoolean inequalities by:

    \begin{AletheX}
        $i$. & \ctxsep & $(\lsymb{bvuge}\ x\ y) \approx A$ & (\currule)
    \end{AletheX}
    The term ``$A$'' is `true' iff:

    \[
        \sum_{i=0}^{n-1} 2^i\mathbf{x}_{i} - \sum_{i=0}^{n-1} 2^i\mathbf{y}_{i} \ge 0.
    \]

\end{RuleDescription}

\begin{RuleDescription}{pbblast_bvule}
    The `unsigned-less-or-equal' operation over BitVectors with $n$ bits is expressed using PseudoBoolean inequalities by:

    \begin{AletheX}
        $i$. & \ctxsep & $(\lsymb{bvule}\ x\ y) \approx A$ & (\currule)
    \end{AletheX}
    The term ``$A$'' is `true' iff:

    \[
        \sum_{i=0}^{n-1} 2^i\mathbf{y}_{i} - \sum_{i=0}^{n-1} 2^i\mathbf{x}_{i} \ge 0.
    \]

    \noindent
    Or in terms of \proofRule{pbblast_bvuge}:

    \begin{AletheX}
        $i$. & \ctxsep & $(\lsymb{bvule}\ x\ y) \approx (\lsymb{bvuge}\ y\ x)$ & (\currule)
    \end{AletheX}
\end{RuleDescription}

\begin{RuleDescription}{pbblast_bvslt}
    The `signed-less-than' operation over BitVectors with $n$ bits is expressed using PseudoBoolean inequalities by:

    \begin{AletheX}
        $i$. & \ctxsep & $(\lsymb{bvslt}\ x\ y) \approx A$ & (\currule)
    \end{AletheX}
    The term ``$A$'' is `true' iff:

    \[
        -(2^{n-1})\mathbf{y}_{n-1} + \sum_{i=0}^{n-2} 2^i\mathbf{y}_{i} + 2^{n-1} \mathbf{x}_{n-1} - \sum_{i=0}^{n-2} 2^i\mathbf{x}_{i} \geq 1
    \]

\end{RuleDescription}

\begin{RuleDescription}{pbblast_bvsgt}
    The `signed-greater-than' operation over BitVectors with $n$ bits is expressed using PseudoBoolean inequalities by:

    \begin{AletheX}
        $i$. & \ctxsep & $(\lsymb{bvsgt}\ x\ y) \approx A$ & (\currule)
    \end{AletheX}
    The term ``$A$'' is `true' iff:

    \[
        -(2^{n-1})\mathbf{x}_{n-1} + \sum_{i=0}^{n-2} 2^i\mathbf{x}_{i} + 2^{n-1} \mathbf{y}_{n-1} - \sum_{i=0}^{n-2} 2^i\mathbf{y}_{i} \geq 1
    \]

    \noindent
    Or in terms of \proofRule{pbblast_bvslt}:

    \begin{AletheX}
        $i$. & \ctxsep & $(\lsymb{bvsgt}\ x\ y) \approx (\lsymb{bvslt}\ y\ x)$ & (\currule)
    \end{AletheX}
\end{RuleDescription}

\begin{RuleDescription}{pbblast_bvsge}
    The `signed-greater-or-equal' operation over BitVectors with $n$ bits is expressed using PseudoBoolean inequalities by:

    \begin{AletheX}
        $i$. & \ctxsep & $(\lsymb{bvsge}\ x\ y) \approx A$ & (\currule)
    \end{AletheX}
    The term ``$A$'' is `true' iff:

    \[
        -(2^{n-1})\mathbf{x}_{n-1} + \sum_{i=0}^{n-2} 2^i\mathbf{x}_{i} + 2^{n-1}\mathbf{y}_{n-1} - \sum_{i=0}^{n-2} 2^i\mathbf{y}_{i} \geq 0
    \]

\end{RuleDescription}

\begin{RuleDescription}{pbblast_bvsle}
    The `signed-less-or-equal' operation over BitVectors with $n$ bits is expressed using PseudoBoolean inequalities by:

    \begin{AletheX}
        $i$. & \ctxsep & $(\lsymb{bvsle}\ x\ y) \approx A$ & (\currule)
    \end{AletheX}
    The term ``$A$'' is `true' iff:

    \[
        -(2^{n-1})\mathbf{y}_{n-1} + \sum_{i=0}^{n-2} 2^i\mathbf{y}_{i} + 2^{n-1}\mathbf{x}_{n-1} - \sum_{i=0}^{n-2} 2^i\mathbf{x}_{i} \geq 0
    \]

    \noindent
    Or in terms of \proofRule{pbblast_bvsge}:

    \begin{AletheX}
        $i$. & \ctxsep & $(\lsymb{bvsle}\ x\ y) \approx (\lsymb{bvsge}\ y\ x)$ & (\currule)
    \end{AletheX}
\end{RuleDescription}

\begin{RuleDescription}{pbblast_pbbvar}
    Conversion from a BitVector of $n$ bits to $n$ PseudoBoolean variables passed to pbbT:

    \begin{AletheX}
        $i$. & \ctxsep & $x \approx \lsymb{pbbT}\; x_1 \dots x_{n+1}$ & (\currule)
    \end{AletheX}
\end{RuleDescription}

\begin{RuleDescription}{pbblast_pbbconst}
    Constraints on each bit of the constant BitVector b:

    \begin{AletheX}
        $i$. & \ctxsep & $\left(b \approx \lsymb{pbbT}\ r\right) \land \bigwedge_{i=0}^{n-1}{\left(r_i = \lsymb{PB\_ZERO\_OR\_ONE}(b_{n-i-1})\right)}$ & (\currule) \\
    \end{AletheX}
    % TODO: Explain PB_ZERO_OR_ONE := if b_i is 1 then b_i = 1 else b_i = 0
    \noindent
    In which we expand \textbf{PB\_ZERO\_OR\_ONE($b_i$)} into:
    \begin{itemize}
        \item $\left(b_i = 0\right)$ if $b_i$ is $0$
        \item $\left(b_i = 1\right)$ if $b_i$ is $1$
    \end{itemize}
\end{RuleDescription}

\begin{RuleDescription}{pbblast_bvxor}
    The `bvxor' operation over BitVectors with $n$ bits is expressed using PseudoBoolean inequalities by:

    \begin{AletheX}
        $i$. & \ctxsep & $(\lsymb{bvxor}\ x\ y) \approx [r_0,\dots,r_{n-1}] \land A$ & (\currule) \\
    \end{AletheX}
    The term ``$A$'' is the conjunction of these PseudoBoolean inequalities and the term \textbf{r} stands
    for the result of the `bvxor' operation between \textbf{x} and \textbf{y}, for $0 \le i < n$:

    \[ -\textbf{r}_i+\textbf{x}_i+\textbf{y}_i\ge  0 \]
    \[ -\textbf{r}_i-\textbf{x}_i-\textbf{y}_i\ge -2 \]
    \[  \textbf{r}_i+\textbf{x}_i-\textbf{y}_i\ge  0 \]
    \[  \textbf{r}_i-\textbf{x}_i+\textbf{y}_i\ge  0 \]
\end{RuleDescription}

% ADD
% OR
% NOT


\begin{RuleDescription}{pbblast_bvand}
    The `bvand' operation over BitVectors with $n$ bits is expressed using PseudoBoolean inequalities by:

    \begin{AletheX}
        $i$. & \ctxsep & $(\lsymb{bvand}\ x\ y) \approx [r_0,\dots,r_{n-1}] \land A$ & (\currule) \\
    \end{AletheX}
    The term ``$A$'' is the conjunction of these PseudoBoolean inequalities and the term \textbf{r} stands
    for the result of the `bvand' operation between \textbf{x} and \textbf{y}, for $0 \le i < n$:

    \[ \textbf{x}_i-\textbf{r}_i\ge 0 \]
    \[ \textbf{y}_i-\textbf{r}_i\ge 0 \]
    \[ \textbf{r}_i-\textbf{x}_i-\textbf{y}_1\ge -1 \]
\end{RuleDescription}

\newpage
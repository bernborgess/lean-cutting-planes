\begin{RuleDescription}{cp_addition}
    \begin{AletheX}
        $i$. & \ctxsep & $\varphi$ & \currule \\
    \end{AletheX}
    TODO where $\varphi$ corresponds up to the orientation of equalities
    to a formula asserted in the input problem, or $\varphi$ is a local assumption
    in a subproof.
    \ruleparagraph{Remark.}
    This rule can not be used by the
    \inlineAlethe{(step }\dots\inlineAlethe{)} command. Instead it corresponds to the dedicated
    \inlineAlethe{(assume }\dots\inlineAlethe{)} command.
\end{RuleDescription}


\begin{RuleDescription}{cp_multiplication}
    \begin{AletheX}
        $i$. & \ctxsep & $\varphi$ & \currule \\
    \end{AletheX}
    TODO where $\varphi$ corresponds up to the orientation of equalities
    to a formula asserted in the input problem, or $\varphi$ is a local assumption
    in a subproof.
    \ruleparagraph{Remark.}
    This rule can not be used by the
    \inlineAlethe{(step }\dots\inlineAlethe{)} command. Instead it corresponds to the dedicated
    \inlineAlethe{(assume }\dots\inlineAlethe{)} command.
\end{RuleDescription}


\begin{RuleDescription}{cp_divison}
    \begin{AletheX}
        $i$. & \ctxsep & $\varphi$ & \currule \\
    \end{AletheX}
    TODO where $\varphi$ corresponds up to the orientation of equalities
    to a formula asserted in the input problem, or $\varphi$ is a local assumption
    in a subproof.
    \ruleparagraph{Remark.}
    This rule can not be used by the
    \inlineAlethe{(step }\dots\inlineAlethe{)} command. Instead it corresponds to the dedicated
    \inlineAlethe{(assume }\dots\inlineAlethe{)} command.
\end{RuleDescription}


\begin{RuleDescription}{cp_saturation}
    \begin{AletheX}
        $i$. & \ctxsep & $\varphi$ & \currule \\
    \end{AletheX}
    TODO where $\varphi$ corresponds up to the orientation of equalities
    to a formula asserted in the input problem, or $\varphi$ is a local assumption
    in a subproof.
    \ruleparagraph{Remark.}
    This rule can not be used by the
    \inlineAlethe{(step }\dots\inlineAlethe{)} command. Instead it corresponds to the dedicated
    \inlineAlethe{(assume }\dots\inlineAlethe{)} command.
\end{RuleDescription}



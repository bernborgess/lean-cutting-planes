\begin{RuleDescription}{cp_addition}
    A step of the \currule{} rule represents a step in the cutting planes reasoning, that works on
    Pseudo-Boolean Constraints, that are 0-1 integer linear inequalities.

    A pseudo boolean inequality is a term of the form
    \[
        \sum_i{a_i l_i} \ge A
    \]
    where we have:
    \begin{itemize}
        \item constant $A \in \mathbb{N}$
        \item coefficients $ a_i \in \mathbb{N}$
        \item literals $ l_i: x_i\ \text{or}\ \overline{x_i}\ (\text{where } x_i + \overline{x_i} = 1)$
        \item variables $ x_i $ take values $ 0 = false $ or $ 1 = true$
    \end{itemize}

    In alethe such formula will be defined as:

    \centerline{\inlineAlethe{(>= (+ <TERMS> 0) <A>)}}
    \hfill

    where \inlineAlethe{<TERMS>} is a list of either:
    \begin{enumerate}
        \item \inlineAlethe{(* <a_i> <x_i>)} in a plain literal $a_i x_i$.
        \item \inlineAlethe{(* (- 1 <a_i>) <x_i>)} in a negated literal $a_i \overline x_i$
    \end{enumerate}

    and \inlineAlethe{<A>} is the natural constant $A$.

    Using this format, two constraints can be added together, adding the coefficients and the
    constants. Another behaviour is that $x_i$ and $\overline x_i$ that may appear together
    cancel each other:

    \begin{AletheS}
        $i_1$. & \ctxsep & \, & ${\sum_i{a_i l_i} \ge A}$  \\
        $i_2$. & \ctxsep  & \, & ${\sum_i{b_i l_i} \ge B}$ \\
        $j$. & \ctxsep  & \, & ${\sum_i{(a_i + b_i) l_i} \ge (A+B)}$  & (\currule\; $i_1$, $i_2$)
    \end{AletheS}

\end{RuleDescription}

\newpage

\begin{RuleDescription}{cp_addition2}
    A step of the \currule{} rule represents the addition of two pseudo-Boolean
    constraints using cutting planes reasoning. Pseudo-Boolean constraints are
    0-1 integer linear inequalities of the form:
    \[
        \sum_i a_i \times l_i \geq A
    \]
    where $A \in \mathbb{N}$ is a constant, $a_i \in \mathbb{N}$ are coefficients,
    and $l_i$ are literals of the form $x_i$ or $\overline{x_i}$ (where $x_i + \overline{x_i} = 1$),
    with $x_i$ taking pseudo-boolean values $0$ (false) or $1$ (true).

    The \proofRule{cp_addition2} rule allows two pseudo-Boolean constraints to be
    added together, combining their coefficients and constants. Literals $x_i$ and
    $\overline{x_i}$ that appear together cancel each other.

    Formally, given two constraints:
    \[
        \sum_i a_i \times l_i \geq A \quad \text{and} \quad \sum_i b_i \times l_i \geq B,
    \]
    the result of applying the \currule{} rule is:
    \[
        \sum_i (a_i + b_i) \times l_i \geq (A + B).
    \]

    A {\currule} step in the proof has the form:

    \begin{AletheS}
        $i_1$. & \ctxsep & \, & ${\sum_i{a_i l_i} \ge A}$  \\
        $i_2$. & \ctxsep  & \, & ${\sum_i{b_i l_i} \ge B}$ \\
        $j$. & \ctxsep  & \, & ${\sum_i{(a_i + b_i) l_i} \ge (A+B)}$  & (\currule\; $i_1$, $i_2$)
    \end{AletheS}

    To verify a \proofRule{cp_addition2} step, one must check that the two given
    pseudo-Boolean constraints are valid and that their combination satisfies
    the addition rule.

\end{RuleDescription}

\begin{RuleExample}
    A simple \proofRule{cp_addition2} step might look like this:

    \begin{AletheVerb}
        (assume c1 (>= (+ (* 1 x1) 0) 1))
        (assume c2 (>= (+ (* 1 x2) 0) 1))
        (step t1 (cl (>= (+ (* 1 x1) (* 1 x2) 0) 2))
        :rule cp_addition :premises (c1 c2))
    \end{AletheVerb}

    In this example, we are adding two constraints.
    \[
        x_1 \geq 1 \quad \text{and} \quad x_2 \geq 1.
    \]
    After applying the \proofRule{cp_addition2} rule,
    the combined constraint is:
    \[
        x_1 + x_2 \geq 1
    \]
\end{RuleExample}

\begin{RuleExample}
    This \proofRule{cp_addition2} example has negated literals that cancel each other:

    \begin{AletheVerb}
        (assume c1 (>= (+ (* 2 x1) (* 3 x2) 0) 2))
        (assume c2 (>= (+ (* 1 (- 1 x1)) (* 3 (- 1 x2)) 0) 4))
        (step t1 (cl (>= (+ (* 1 x1) 0) 2))
        :rule cp_addition :premises (c1 c2))
    \end{AletheVerb}

    In this example, we are adding two constraints.
    \[
        2 x_1 + 3 x_2 \geq 2 \quad \text{and} \quad \overline{x_1} + 3 \overline{x_2} \geq 4.
    \]
    After applying the \proofRule{cp_addition2} rule,
    the combined constraint is:
    \[
        x_1 \geq 2
    \]
    After simplification, this results in a contradiction.
\end{RuleExample}



\begin{RuleDescription}{cp_multiplication}
    A constraint can be multiplied by any $c \in \mathbb{N}^+$:

    \begin{AletheS}
        $i$. & \ctxsep & \, & ${\sum_i{a_i l_i} \ge A}$  \\
        $j$. & \ctxsep  & \, & ${\sum_i{c a_i l_i} \ge c A}$  & (\currule\; $i$)\,[$c$]
    \end{AletheS}

\end{RuleDescription}


\begin{RuleDescription}{cp_divison}
    A constraint can be divided by any $c \in \mathbb{N}^+$,
    and the the ceiling of this division in applied:

    \begin{AletheS}
        $i$. & \ctxsep & \, & ${\sum_i{a_i l_i} \ge A}$  \\
        $j$. & \ctxsep  & \, & ${\sum_i{ \lceil \frac{a_i}{c} \rceil l_i} \ge \lceil \frac{A}{c} \rceil}$  & (\currule\; $i$)\,[$c$]
    \end{AletheS}

\end{RuleDescription}


\begin{RuleDescription}{cp_saturation}
    A constraint can replace its coefficients by the minimum between them and the constant:

    \begin{AletheS}
        $i$. & \ctxsep & \, & ${\sum_i{a_i l_i} \ge A}$  \\
        $j$. & \ctxsep  & \, & ${\sum_i{ \min(a_i,A)\cdot l_i} \ge A}$  & (\currule\; $i$)
    \end{AletheS}

\end{RuleDescription}

\newpage
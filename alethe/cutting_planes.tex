\begin{RuleDescription}{cp_addition}
    A step of the \currule{} rule represents a step in the cutting planes reasoning, that works on
    Pseudo-Boolean Constraints, that are 0-1 integer linear inequalities.

    A pseudo boolean inequality is a term of the form
    \[
        \sum_i{a_i l_i} \ge A
    \]
    where we have:
    \begin{itemize}
        \item constant $A \in \mathbb{N}$
        \item coefficients $ a_i \in \mathbb{N}$
        \item literals $ l_i: x_i\ \text{or}\ \overline{x_i}\ (\text{where } x_i + \overline{x_i} = 1)$
        \item variables $ x_i $ take values $ 0 = false $ or $ 1 = true$
    \end{itemize}

    In alethe such formula will be defined as:

    \centerline{\inlineAlethe{(>= (+ <TERMS> 0) <A>)}}
    \hfill

    where \inlineAlethe{<TERMS>} is a list of either:
    \begin{enumerate}
        \item \inlineAlethe{(* <a_i> <x_i>)} in a plain literal $a_i x_i$.
        \item \inlineAlethe{(* (- 1 <a_i>) <x_i>)} in a negated literal $a_i \overline x_i$
    \end{enumerate}

    and \inlineAlethe{<A>} is the natural constant $A$. \\

    Using this format, two constraints can be added together, adding the coefficients and the
    constants. Another behaviour is that $x_i$ and $\overline x_i$ that may appear together
    cancel each other:

    \begin{AletheXS}
        $j$. & ${\sum_i{a_i l_i} \ge A}$ & ${\sum_i{b_i l_i} \ge B}$ \\
        \spsep
        $k$. & $\Gamma$ & \ctxsep  & ${\sum_i{(a_i + b_i) l_i} \ge (A+B)}$ & \currule{} \\
    \end{AletheXS}

    \textit{TODO: make the above line fit the "rule" format appropriately}

\end{RuleDescription}


\begin{RuleDescription}{cp_multiplication}
    \begin{AletheX}
        $i$. & \ctxsep & $\varphi$ & \currule \\
    \end{AletheX}
    TODO
\end{RuleDescription}


\begin{RuleDescription}{cp_divison}
    \begin{AletheX}
        $i$. & \ctxsep & $\varphi$ & \currule \\
    \end{AletheX}
    TODO
\end{RuleDescription}


\begin{RuleDescription}{cp_saturation}
    \begin{AletheX}
        $i$. & \ctxsep & $\varphi$ & \currule \\
    \end{AletheX}
    TODO
\end{RuleDescription}



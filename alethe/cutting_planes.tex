\begin{RuleDescription}{cp_addition}
    A step of the \currule{} rule represents the addition of two pseudo-Boolean
    constraints using cutting planes reasoning. Pseudo-Boolean constraints are
    0-1 integer linear inequalities of the form:
    \[
        \sum_i a_i \times l_i \geq A
    \]
    where $A$ is called \textbf{constant}, $a_i$ are \textbf{coefficients},
    and $l_i$ are \textbf{literals}, that are either:
    \begin{itemize}
        \item \textbf{plain} literal, a term \inlineAlethe{x};
        \item \textbf{negated} literal, a term of the form \inlineAlethe{(- 1 x)}
    \end{itemize}

    where the \inlineAlethe{x} value is a pseudo-boolean variable,
    i.e. it will resolve to values \inlineAlethe{0} (false) or \inlineAlethe{1} (true).
    All these values are of sort \inlineAlethe{Int}.

    To form a summation we use a list of added terms of form,
    \inlineAlethe{(+ <T1> <T2> ... 0)} and each term is
    \inlineAlethe{(* a_i <L1>)}, with a coefficient and a literal.\\

    The \currule{} rule allows two pseudo-Boolean constraints to be
    added together, combining their coefficients and constants. Negated and plain literals
    over the same variable cancel each other.

    Formally, given two constraints:
    \[
        \sum_i a_i \times l_i \geq A \quad \text{and} \quad \sum_i b_i \times l_i \geq B,
    \]
    the result of applying the \currule{} rule is:
    \[
        \sum_i (a_i + b_i) \times l_i \geq (A + B).
    \]

    A {\currule} step in the proof has the form:

    \begin{AletheS}
        $i_1$. & \ctxsep & \, & ${\sum_i{a_i l_i} \ge A}$  \\
        $i_2$. & \ctxsep  & \, & ${\sum_i{b_i l_i} \ge B}$ \\
        $j$. & \ctxsep  & \, & ${\sum_i{(a_i + b_i) l_i} \ge (A+B)}$  & (\currule\; $i_1$, $i_2$)
    \end{AletheS}

    To verify a \currule{} step, one must check that the two given
    pseudo-Boolean constraints are valid and that their combination satisfies
    the addition rule.

\end{RuleDescription}

\begin{RuleExample}
    A simple \proofRule{cp_addition} step might look like this:

    \begin{AletheVerb}
        (assume c1 (>= (+ (* 1 x1) 0) 1))
        (assume c2 (>= (+ (* 1 x2) 0) 1))
        (step t1 (cl (>= (+ (* 1 x1) (* 1 x2) 0) 2))
        :rule cp_addition :premises (c1 c2))
    \end{AletheVerb}

    In this example, we are adding two constraints.
    \[
        x_1 \geq 1 \quad \text{and} \quad x_2 \geq 1.
    \]
    After applying the \proofRule{cp_addition} rule,
    the combined constraint is:
    \[
        x_1 + x_2 \geq 1
    \]
\end{RuleExample}

\begin{RuleExample}
    This \proofRule{cp_addition} example has negated literals that cancel each other:

    \begin{AletheVerb}
        (assume c1 (>= (+ (* 2 x1) (* 3 x2) 0) 2))
        (assume c2 (>= (+ (* 1 (- 1 x1)) (* 3 (- 1 x2)) 0) 4))
        (step t1 (cl (>= (+ (* 1 x1) 0) 2))
        :rule cp_addition :premises (c1 c2))
    \end{AletheVerb}

    In this example, we are adding two constraints.
    \[
        2 x_1 + 3 x_2 \geq 2 \quad \text{and} \quad \overline{x_1} + 3 \overline{x_2} \geq 4.
    \]
    After applying the \proofRule{cp_addition} rule,
    the combined constraint is:
    \[
        x_1 \geq 2
    \]
    After simplification, this results in a contradiction.
\end{RuleExample}



\begin{RuleDescription}{cp_multiplication}
    A constraint can be multiplied by any $c \in \mathbb{N}^+$:

    \begin{AletheS}
        $i$. & \ctxsep & \, & ${\sum_i{a_i l_i} \ge A}$  \\
        $j$. & \ctxsep  & \, & ${\sum_i{c a_i l_i} \ge c A}$  & (\currule\; $i$)\,[$c$]
    \end{AletheS}

\end{RuleDescription}


\begin{RuleDescription}{cp_divison}
    A constraint can be divided by any $c \in \mathbb{N}^+$,
    and the the ceiling of this division in applied:

    \begin{AletheS}
        $i$. & \ctxsep & \, & ${\sum_i{a_i l_i} \ge A}$  \\
        $j$. & \ctxsep  & \, & ${\sum_i{ \lceil \frac{a_i}{c} \rceil l_i} \ge \lceil \frac{A}{c} \rceil}$  & (\currule\; $i$)\,[$c$]
    \end{AletheS}

\end{RuleDescription}


\begin{RuleDescription}{cp_saturation}
    A constraint can replace its coefficients by the minimum between them and the constant:

    \begin{AletheS}
        $i$. & \ctxsep & \, & ${\sum_i{a_i l_i} \ge A}$  \\
        $j$. & \ctxsep  & \, & ${\sum_i{ \min(a_i,A)\cdot l_i} \ge A}$  & (\currule\; $i$)
    \end{AletheS}

\end{RuleDescription}

\newpage
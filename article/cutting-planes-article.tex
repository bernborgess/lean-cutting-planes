\documentclass[conference]{IEEEtran}
\IEEEoverridecommandlockouts
% The preceding line is only needed to identify funding in the first footnote. If that is unneeded, please comment it out.
\usepackage{cite}
\usepackage{amsmath,amssymb,amsfonts}
\usepackage{algorithmic}
\usepackage{graphicx}
\usepackage{textcomp}
\usepackage{xcolor}
\def\BibTeX{{\rm B\kern-.05em{\sc i\kern-.025em b}\kern-.08em
    T\kern-.1667em\lower.7ex\hbox{E}\kern-.125emX}}
\begin{document}

\title{Modelando e verificando o cálculo formal de Planos de Corte com Lean 4\\
{\footnotesize Projeto Orientadado em Computação I - Pesquisa Científica}
}

\author{\IEEEauthorblockN{Bernardo Borges}
    \IEEEauthorblockA{\textit{Departamento de Ciência da Computação} \\
        \textit{Universidade Federal de Minas Gerais}\\
        Belo Horizonte, Brasil\\
        bernardoborges@dcc.ufmg.br}
}

\maketitle

\begin{abstract}
    Este documento ?
\end{abstract}

\begin{IEEEkeywords}
    lean, cutting planes, formal methods, pseudo boolean reasoning
\end{IEEEkeywords}

\section{Introdução}
Capítulo introdutório: caracterização do problema, motivação, objetivos, breve descrição da estrutura do trabalho.

\section{Referencial Teórico}
Capítulo referencial: identificação de trabalhos correlatos, referencial teórico.

\subsection{Pseudo-Booleans}
O que são PBs?

\section{Contribuições}
Capítulo de contribuição: descrição organizada das atividades conduzidas pelo
aluno.

\section*{Conclusões}
Capítulo de fechamento: conclusões e relação de trabalhos futuros.

\section*{Referencias Bibliográficas}

Referências bibliográficas contidas no texto.

\begin{thebibliography}{00}
    \bibitem{b1} J. Nordström, ``A Unified Proof System for Discrete Combinatorial Problems'', November 2023.
\end{thebibliography}
\vspace{12pt}
\end{document}

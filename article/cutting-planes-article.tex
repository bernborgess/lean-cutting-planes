\documentclass[conference]{IEEEtran}
\IEEEoverridecommandlockouts
% The preceding line is only needed to identify funding in the first footnote. If that is unneeded, please comment it out.
\usepackage{cite}
\usepackage{amsmath,amssymb,amsfonts}
\usepackage{algorithmic}
\usepackage{graphicx}
\usepackage{textcomp}
\usepackage{xcolor}
\usepackage[mathletters]{ucs}
\usepackage[utf8x]{inputenc}
\usepackage{minted}

\def\BibTeX{{\rm B\kern-.05em{\sc i\kern-.025em b}\kern-.08em
    T\kern-.1667em\lower.7ex\hbox{E}\kern-.125emX}}
\begin{document}

\title{Modelando e verificando o cálculo formal de Planos de Corte com Lean 4\\
{\footnotesize Projeto Orientadado em Computação I - Pesquisa Científica}
}

\author{\IEEEauthorblockN{Bernardo Borges}
    \IEEEauthorblockA{\textit{Departamento de Ciência da Computação} \\
        \textit{Universidade Federal de Minas Gerais}\\
        Belo Horizonte, Brasil\\
        bernardoborges@dcc.ufmg.br}
}

\maketitle

\begin{abstract}
    Esta pesquisa envolve a definição e verificação da lógica pseudo-booleana de
    Planos de Corte utilizando o provador de teoremas Lean 4

\end{abstract}

\begin{IEEEkeywords}
    lean, cutting planes, formal methods, pseudo boolean reasoning
\end{IEEEkeywords}

\section{Introdução}

% TODO
% O cálculo formal de Planos de Corte é uma técnica matemática utilizada na área de otimização,
% particularmente em Programação Linear.  
% Capítulo introdutório: caracterização do problema, motivação, objetivos, breve descrição da estrutura do trabalho.
% * problema de utilizar pseudo booleanos para representacao eficiente de problemas SAT
% * motivacao: Ter certeza que solucoes que utilizam os pseudo-booleans estao corretas
% * objetivos: Usar metodos formais para validar a logica pseudo-booleana e provas que a utilizem
% * estrutura do trabalho: biblioteca de codigo em Lean 4 que implementa as regras da logica pseudo-booleana e prova sua
%   corretude. Alem disso, mostrar exemplos de uso dessa biblioteca com o "Toy Example" presente na apresentacao de Jakob.
%   Criacao de uma documentacao em formato mdbook, que estara disponivel para pesquisadores e matematicos que desejem usar
%   a biblioteca.
Lorem ipsum dolor sit amet, consectetur adipiscing elit. Mauris luctus lacus eu varius fermentum. Nullam lacus urna, semper a euismod sed, hendrerit eu nisi. Donec nec risus turpis. Vestibulum placerat justo sit amet aliquam mattis. Class aptent taciti sociosqu ad litora torquent per conubia nostra, per inceptos himenaeos. Sed vulputate vulputate maximus. Sed euismod turpis vitae libero consectetur gravida. Pellentesque eu nibh ipsum. Ut porta tortor vel pharetra euismod. Nulla quis nunc bibendum, condimentum sem a, tempus lectus.

\section{Referencial Teórico}
% Capítulo referencial: identificação de trabalhos correlatos, referencial teórico.

\subsection{Satisfiabilidade}
O problema da satisfabilidade booleana (SAT) é muito importante para a Ciência da Computação, sendo o
primeiro demonstrado ser da class NP-Completo. Ele é o problema de decidir para dada expressão booleana,
se é possível escolher valores para as variáveis de forma que a expressão como um todo seja verdadeira.
Este problema é extensivamente pesquisado na área de métodos formais\cite{SatLive} e existem inclusive competições para
definir qual é o melhor solucionador do mundo \cite{SatComp}.

\subsection{Pseudo-Booleans}
Um formato comumente utilizado para representar expressões booleanas é a Forma Normal Conjuntiva (CNF), que
consiste da conjunção de cláusulas, em que cada cláusula é a disjunção de variáveis ou negação de variáveis\cite{CNF}.
Neste trabalho, introduzimos uma outra representação para expressões, chamados Pseudo-Booleanos, funções estudadas
desde os anos 1960 na área de pesquisa operacional, em programação inteira. Este formato consiste de um somatório
do produto de um coeficiente por um literal, que é maior ou igual a uma constante natural.
Este formato é exponencialmente mais compacto que o CNF, o que motiva seu uso\cite{PBSolve}.

\subsection{Pseudo-Boolean Reasoning}
A lógica formal de Cutting Planes introduz 2 axiomas e 4 regras de inferência, nomeadamente
\textbf{Adição}, \textbf{Multiplicação}, \textbf{Divisão} e \textbf{Saturação},
que permitem derivar novas inequações a partir de outras\cite{CutPlane}.

\subsection{Lean Theorem Prover}
Lean é uma linguagem de programação e provador de teoremas criado por Leonardo de Moura em 2013\cite{LeanProver}, com
influência de ML, Coq e Haskell.
A sua versão mais atual de 2021, Lean 4\cite{Lean4}, se tornou proeminente entre matemáticos, por permitir a
\textbf{formalização} e \textbf{verificação} de teoremas, auxiliando o trabalho teórico.
Em 2021, uma equipe de pesquisadores usou o Lean para verificar a correção de uma prova de Peter Scholze na área
de matemática condensada\cite{LTE}, o que atraiu atenção por formalizar um resultado na vanguarda da pesquisa matemática.
Em 2023, Terence Tao usou o Lean para formalizar uma prova da conjectura Polinomial de Freiman-Ruzsa\cite{PFR},
um resultado publicado por Tao e colaboradores no mesmo ano.

\section{Contribuições}
% Capítulo de contribuição: descrição organizada das atividades conduzidas pelo aluno.
A contribuição deste trabalho está na criação de código em Lean 4 para lidar com a lógica de cutting planes.
Primeiramente, as inequações foram formalmente definidas em Lean, utilizando teoremas e definições de sua biblioteca
matemática mathlib4\cite{mathlib4}. Assim as quatro regras foram definidas e formalizadas, e em seguida um exemplo de
raciocínio, apresentado por Jakob Nordström foi formalizado nessa biblioteca.

% * Definição da inequação pseudo-booleana
\subsection{Definição de Inequações Pseudo-Booleanas}
Uma expressão booleana na forma CNF consiste de:
\begin{equation}
    C_1 \land C_2 \land \dots \land C_n
\end{equation}
onde,
\begin{equation}
    C_i = a_1^i \lor a_2^i \lor \dots \lor a_k^i
\end{equation}
e para cada $j$
\begin{equation}
    a_j^i \in \{ x_1,x_2,\dots,x_m,\neg x_1,\neg x_2,\dots, \neg x_m \}
\end{equation}

Um exemplo desse formato é a formula:
\begin{equation}
    (x_1 \lor x_2) \land (x_1 \lor \neg x_3) \land (x_2 \lor \neg x_3)
\end{equation}
Onde $x_1$, $x_2$ e $x_3$ são nossas variáveis booleanas e a atribuição $x_1=T$, $x_2=T$, $x_3=T$ satisfaz essa fórmula.

O formato pseudo-booleano consiste de:
\begin{equation}
    \sum_i{a_i l_i} \ge A
\end{equation}
onde,
\begin{equation}
    \begin{gathered}
        A, a_i \in \mathbb{N} \\
        l_i \in \{ x_i, \overline x_i \}, \qquad x_i + \overline x_i = 1
    \end{gathered}
\end{equation}

A mesma fórmula acima nesse formato é expressa por:
\begin{equation}
    x_1 + x_2 + \overline x_3 \ge 2
\end{equation}

A implementação em Lean 4 consiste de definir o tipo para coeficientes \textit{Coeff}, onde usamos a estrutura
\textit{FinVec}, que permite termos uma lista com $n$ elementos, onde cada elemento da lista será um par
de dois números naturais, o tipo $\mathbb{N}$ em Lean:
\begin{minted}[fontsize=\small]{lean4}
abbrev Coeff (n : ℕ) := Fin n → (ℕ × ℕ)
\end{minted}

Esse par consiste do coeficiente de $x_i$ no primeiro elemento e o coeficiente de $\overline x_i$ no segundo elemento.

Com essa definição, definimos o \textit{PBSum}, que, com os coeficientes $cs$ e os valores 0-1 $xs$,
utiliza o \textit{BigOperator} de somatório ($\sum$) para somar os elementos da lista iterando no índice $i$:
\begin{minted}[fontsize=\small]{lean4}
def PBSum (cs : Coeff n) (xs : Fin n → Fin 2) :=
  ∑ i, let (p,n) := cs i;
    if xs i = 1 then p else n
\end{minted}


% * Prova da Regra Multiplication
\subsection{Prova da Regra Multiplicação}
Lorem ipsum dolor sit amet, consectetur adipiscing elit. Mauris luctus lacus eu varius fermentum. Nullam lacus urna, semper a euismod sed, hendrerit eu nisi. Donec nec risus turpis. Vestibulum placerat justo sit amet aliquam mattis. Class aptent taciti sociosqu ad litora torquent per conubia nostra, per inceptos himenaeos. Sed vulputate vulputate maximus. Sed euismod turpis vitae libero consectetur gravida. Pellentesque eu nibh ipsum. Ut porta tortor vel pharetra euismod. Nulla quis nunc bibendum, condimentum sem a, tempus lectus.

% * Prova da Regra Saturation
\subsection{Prova da Regra Saturação}
Lorem ipsum dolor sit amet, consectetur adipiscing elit. Mauris luctus lacus eu varius fermentum. Nullam lacus urna, semper a euismod sed, hendrerit eu nisi. Donec nec risus turpis. Vestibulum placerat justo sit amet aliquam mattis. Class aptent taciti sociosqu ad litora torquent per conubia nostra, per inceptos himenaeos. Sed vulputate vulputate maximus. Sed euismod turpis vitae libero consectetur gravida. Pellentesque eu nibh ipsum. Ut porta tortor vel pharetra euismod. Nulla quis nunc bibendum, condimentum sem a, tempus lectus.

% * Prova da Regra Division
\subsection{Prova da Regra Divisão}
Lorem ipsum dolor sit amet, consectetur adipiscing elit. Mauris luctus lacus eu varius fermentum. Nullam lacus urna, semper a euismod sed, hendrerit eu nisi. Donec nec risus turpis. Vestibulum placerat justo sit amet aliquam mattis. Class aptent taciti sociosqu ad litora torquent per conubia nostra, per inceptos himenaeos. Sed vulputate vulputate maximus. Sed euismod turpis vitae libero consectetur gravida. Pellentesque eu nibh ipsum. Ut porta tortor vel pharetra euismod. Nulla quis nunc bibendum, condimentum sem a, tempus lectus.

% * Prova da Regra Addition
\subsection{Prova da Regra Adição}
Lorem ipsum dolor sit amet, consectetur adipiscing elit. Mauris luctus lacus eu varius fermentum. Nullam lacus urna, semper a euismod sed, hendrerit eu nisi. Donec nec risus turpis. Vestibulum placerat justo sit amet aliquam mattis. Class aptent taciti sociosqu ad litora torquent per conubia nostra, per inceptos himenaeos. Sed vulputate vulputate maximus. Sed euismod turpis vitae libero consectetur gravida. Pellentesque eu nibh ipsum. Ut porta tortor vel pharetra euismod. Nulla quis nunc bibendum, condimentum sem a, tempus lectus.

% * Implementação do Toy Example
\subsection{Implementação do Exemplo}
Lorem ipsum dolor sit amet, consectetur adipiscing elit. Mauris luctus lacus eu varius fermentum. Nullam lacus urna, semper a euismod sed, hendrerit eu nisi. Donec nec risus turpis. Vestibulum placerat justo sit amet aliquam mattis. Class aptent taciti sociosqu ad litora torquent per conubia nostra, per inceptos himenaeos. Sed vulputate vulputate maximus. Sed euismod turpis vitae libero consectetur gravida. Pellentesque eu nibh ipsum. Ut porta tortor vel pharetra euismod. Nulla quis nunc bibendum, condimentum sem a, tempus lectus.

\section*{Conclusões}
% Capítulo de fechamento: conclusões e relação de trabalhos futuros.
Lorem ipsum dolor sit amet, consectetur adipiscing elit. Mauris luctus lacus eu varius fermentum. Nullam lacus urna, semper a euismod sed, hendrerit eu nisi. Donec nec risus turpis. Vestibulum placerat justo sit amet aliquam mattis. Class aptent taciti sociosqu ad litora torquent per conubia nostra, per inceptos himenaeos. Sed vulputate vulputate maximus. Sed euismod turpis vitae libero consectetur gravida. Pellentesque eu nibh ipsum. Ut porta tortor vel pharetra euismod. Nulla quis nunc bibendum, condimentum sem a, tempus lectus.

\begin{thebibliography}{00}
    \bibitem{SatLive}       D. Le Berre, ``Sat Live! keep up to date with research on the satisfiability problem'', Acesso em http://www.satlive.org/.
    \bibitem{SatComp}       M. Heule, M. Jävisalo, M. Suda, ``The International SAT Competition Web Page'', Acesso em https://satcompetition.github.io/.
    \bibitem{CNF}           ``Conjunctive normal form'', Encyclopedia of Mathematics, EMS Press, 2001.
    \bibitem{PBSolve}       J. Nordström, ``Pseudo-Boolean Solving and Optimization'', Fevereiro de 2021.
    \bibitem{CutPlane}      J. Nordström, ``A Unified Proof System for Discrete Combinatorial Problems'', Novembro de 2023.
    \bibitem{LeanProver}    L. de Moura, S. Kong, J. Avigad, F. van Doorn, J. von Raumer ``The Lean Theorem Prover'', 25th International Conference on Automated Deduction (CADE-25), Berlin, Germany, 2015. Acesso em https://lean-lang.org/papers/system.pdf.
    \bibitem{Lean4}         L. de Moura, S. Ullrich ``The Lean 4 Theorem Prover and Programming Language'', 28th International Conference on Automated Deduction (CADE-28), Pittsburgh, USA, 2021. Acesso em https://lean-lang.org/papers/lean4.pdf.
    \bibitem{LTE}           J. Commelin, P. Scholze ``Liquid Tensor Experiment''. Acesso em https://math.commelin.net/files/LTE.pdf
    \bibitem{PFR}           T. Tao, ``The Polynomial Freiman-Ruzsa Conjecture'', Novembro de 2023. Acesso em https://teorth.github.io/pfr/.
    \bibitem{mathlib4}      The mathlib Community. 2020. ``The lean mathematical library''. In CPP 2020. 367-381. https://doi.org/10.1145/3372885.3373824
\end{thebibliography}

\vspace{12pt}

\newpage

\section{Apêndice}

\subsection{Definição de Pseudo-Boolean}

% \begin{minted}{lean4}

% open FinVec BigOperators

% -- TODO: Add support for unicode chars in LaTeX
% abbrev Coeff (n : .) := Fin n → (. . .)


% def PBSum (cs : Coeff n) (xs : Fin n → Fin 2) :=
%   . i, let (p,n) := cs i;
%     if xs i = 1 then p else n

% def PBIneq (cs : Coeff n) (xs : Fin n → Fin 2) (const : .) :=
%   PBSum cs xs . const

% \end{minted}


\end{document}

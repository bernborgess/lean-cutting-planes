\documentclass[conference]{IEEEtran}
\IEEEoverridecommandlockouts
% The preceding line is only needed to identify funding in the first footnote. If that is unneeded, please comment it out.
\usepackage{cite}
\usepackage{amsmath,amssymb,amsfonts}
\usepackage{algorithmic}
\usepackage{graphicx}
\usepackage{textcomp}
\usepackage{xcolor}
\def\BibTeX{{\rm B\kern-.05em{\sc i\kern-.025em b}\kern-.08em
    T\kern-.1667em\lower.7ex\hbox{E}\kern-.125emX}}
\begin{document}

\title{Modelando e verificando o cálculo formal de Planos de Corte com Lean 4\\
{\footnotesize Projeto Orientadado em Computação I - Pesquisa Científica}
}

\author{\IEEEauthorblockN{Bernardo Borges}
    \IEEEauthorblockA{\textit{Departamento de Ciência da Computação} \\
        \textit{Universidade Federal de Minas Gerais}\\
        Belo Horizonte, Brasil\\
        bernardoborges@dcc.ufmg.br}
}

\maketitle

\begin{abstract}
    Esta pesquisa envolve a definição e verificação da lógica pseudo-booleana de
    Planos de Corte utilizando o provador de teoremas Lean 4

\end{abstract}

\begin{IEEEkeywords}
    lean, cutting planes, formal methods, pseudo boolean reasoning
\end{IEEEkeywords}

\section{Introdução}
O problema da satisfabilidade booleana (SAT) é muito importante para a Ciência da Computação, sendo o
primeiro demonstrado ser da class NP-Completo. Ele é o problema de decidir para dada expressão booleana, é
possível escolher valores para as variáveis de forma que a expressão como um todo seja verdadeira.
Este problema é extensivamente pesquisado na área de métodos formais\cite{SatLive} e existem inclusive competições para
definir qual é o melhor solucionador do mundo \cite{SatComp}.

Um formato comumente utilizado para representar essas expressões é a Forma Normal Conjuntiva (CNF), que
consiste da conjunção de cláusulas, em que cada cláusula é a disjunção de variáveis ou negação de variáveis.
% Formula CNF aqui
Entretando, introduzimos uma outra representação para expressões, chamados Pseudo-Booleanos, funções estudadas
desde os anos 1960 na área de pesquisa operacional.
% Formula PB aqui
Este formato é exponencialmente mais compacto que o CNF, o que motiva seu uso\cite{PBSolve}.

% TODO: https://youtu.be/xoAov5RLcis?feature=shared&t=165


O cálculo formal de Planos de Corte é uma técnica matemática utilizada na área de otimização,
particularmente em Programação Linear que
% Capítulo introdutório: caracterização do problema, motivação, objetivos, breve descrição da estrutura do trabalho.

\section{Referencial Teórico}
Capítulo referencial: identificação de trabalhos correlatos, referencial teórico.

\subsection{Pseudo-Booleans}
O que são PBs?

\section{Contribuições}
Capítulo de contribuição: descrição organizada das atividades conduzidas pelo
aluno.

\section*{Conclusões}
Capítulo de fechamento: conclusões e relação de trabalhos futuros.

\begin{thebibliography}{00}
    \bibitem{SatLive}   ``Sat Live! keep up to date with research on the satisfiability problem'', access at http://www.satlive.org/
    \bibitem{SatComp}   ``The International SAT Competition Web Page'', access at https://satcompetition.github.io/.
    \bibitem{PBSolve}   J. Nordström, ``Pseudo-Boolean Solving and Optimization'', February 2021.
    \bibitem{CutPlane}  J. Nordström, ``A Unified Proof System for Discrete Combinatorial Problems'', November 2023.

\end{thebibliography}
\vspace{12pt}
\end{document}

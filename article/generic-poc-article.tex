\documentclass[conference]{IEEEtran}
\IEEEoverridecommandlockouts
\usepackage{algorithmic}
\usepackage{amsmath,amssymb,amsfonts}
\usepackage{cite}
\usepackage{graphicx}
\usepackage{textcomp}
\usepackage{xcolor}
\usepackage[mathletters]{ucs}
\usepackage[utf8x]{inputenc}
\usepackage{minted}

\def\BibTeX{{\rm B\kern-.05em{\sc i\kern-.025em b}\kern-.08em
    T\kern-.1667em\lower.7ex\hbox{E}\kern-.125emX}}
\begin{document}

\title{Generic Poc Article Title Here\\
{\footnotesize Projeto Orientado em Computação I - Pesquisa Científica}
}

\author{\IEEEauthorblockN{Generic Name}
    \IEEEauthorblockA{\textit{Departamento de Ciência da Computação} \\
        \textit{Universidade Federal de Minas Gerais}\\
        Belo Horizonte, Brasil\\
        generic@dcc.ufmg.br}
}

\maketitle

\begin{abstract}
    % TODO
    Esta pesquisa envolve a generic abstract here.
\end{abstract}

\begin{IEEEkeywords}
    formal methods, proof checking
\end{IEEEkeywords}

\section{Introdução}
Caracterização do problema, motivação, objetivos, breve
descrição da estrutura do trabalho.
Citação do Carcara \cite{Carcara}.

\section{Referencial Teórico}
% Capítulo referencial:
Identificação de trabalhos correlatos, referencial teórico.

\section{Contribuições}
% Capítulo de contribuição:
Descrição organizada das atividades conduzidas pelo aluno.

\section*{Conclusões}
% Capítulo de fechamento:
Conclusões e relação de trabalhos futuros.

\begin{thebibliography}{00}
    \bibitem{Carcara}       B. Andreotti, H. Lachnitt, H. Barbosa ``Carcara: An Efficient Proof Checker and Elaborator for SMT Proofs in the Alethe Format'', Abril de 2023, Acesso em https://link.springer.com/chapter/10.1007/978-3-031-30823-9\_19.
\end{thebibliography}

\newpage
\onecolumn
\appendix

\subsection{Generic appendix}
This is appendix text.

\end{document}
